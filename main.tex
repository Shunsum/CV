%%%%%%%%%%%%%%%%%
% This is an example CV created using altacv.cls (v1.1.4, 27 July 2018) written by
% LianTze Lim (liantze@gmail.com), based on the
% Cv created by BusinessInsider at http://www.businessinsider.my/a-sample-resume-for-marissa-mayer-2016-7/?r=US&IR=T
%
%% It may be distributed and/or modified under the
%% conditions of the LaTeX Project Public License, either version 1.3
%% of this license or (at your option) any later version.
%% The latest version of this license is in
%%    http://www.latex-project.org/lppl.txt
%% and version 1.3 or later is part of all distributions of LaTeX
%% version 2003/12/01 or later.
%%%%%%%%%%%%%%%%

%% If you want to use \orcid or the
%% academicons icons, add "academicons"
%% to the \documentclass options.
%% Then compile with XeLaTeX or LuaLaTeX.
% \documentclass[10pt,a4paper,academicons]{altacv}

%% Use the "normalphoto" option if you want a normal photo instead of cropped to a circle
% \documentclass[10pt,a4paper,normalphoto]{altacv}

\documentclass[10pt,a4paper]{altacv}

%% AltaCV uses the fontawesome and academicon fonts
%% and packages.
%% See texdoc.net/pkg/fontawecome and http://texdoc.net/pkg/academicons for full list of symbols.
%% When using the "academicons" option,
%% Compile with LuaLaTeX for best results. If you
%% want to use XeLaTeX, you may need to install
%% Academicons.ttf in your operating system's font %% folder.


% Change the page layout if you need to
\geometry{left=1cm,right=9cm,marginparwidth=6.8cm,marginparsep=1.2cm,top=1cm,bottom=1cm}

% Change the font if you want to.

% If using pdflatex:
\usepackage[utf8]{inputenc}
\usepackage[T1]{fontenc}
\usepackage[default]{lato}

% If using xelatex or lualatex:
% \setmainfont{Lato}

% Change the colours if you want to
\definecolor{VividPurple}{HTML}{008080}
\definecolor{SlateGrey}{HTML}{2E2E2E}
\definecolor{LightGrey}{HTML}{666666}
\colorlet{heading}{VividPurple}
\colorlet{accent}{VividPurple}
\colorlet{emphasis}{SlateGrey}
\colorlet{body}{LightGrey}

% Change the bullets for itemize and rating marker
% for \cvskill if you want to
\renewcommand{\itemmarker}{{\small\textbullet}}
\renewcommand{\ratingmarker}{\faCircle}

%% sample.bib contains your publications
% \addbibresource{sample.bib}

\begin{document}
\name{Xinxin Liu (Shunsum Lau)}
\tagline{Master's Student in Physical Chemistry}
% Cropped to square from https://en.wikipedia.org/wiki/Marissa_Mayer#/media/File:Marissa_Mayer_May_2014_(cropped).jpg, CC-BY 2.0
\photo{2.5cm}{shunsum}
\personalinfo{%
  % Not all of these are required!
  % You can add your own with \printinfo{symbol}{detail}
  \email{shunsumlau@qq.com}
  \phone{+86 199 2753 0284}
  \location{Guangzhou, China}
% \homepage{https://yogs447.blogspot.com/}
% \linebreak
% \twitter{@Majeed\_here}
% \linkedin{www.linkedin.com/in/muhammad-majeed-82475030/}
  \github{github.com/Shunsum} % I'm just making this up though.
%   \orcid{orcid.org/0000-0000-0000-0000} % Obviously making this up too. If you want to use this field (and also other academicons symbols), add "academicons" option to \documentclass{altacv}
}

%% Make the header extend all the way to the right, if you want.
\begin{fullwidth}
\makecvheader
\end{fullwidth}

%% Depending on your tastes, you may want to make fonts of itemize environments slightly smaller
\AtBeginEnvironment{itemize}{\small}

%% Provide the file name containing the sidebar contents as an optional parameter to \cvsection.
%% You can always just use \marginpar{...} if you do
%% not need to align the top of the contents to any
%% \cvsection title in the "main" bar.
\cvsection[page1sidebar]{Education}
\cvevent{MSc in Physical Chemistry}{MOE Key Laboratory of Theoretical Chemistry of Environment;\linebreak School of Environment, South China Normal University}{2022 -- 2025}{Guangzhou, China} 
\textsc{CGPA}: 3.99/5.00
\\
\cvevent{BSc in Applied Chemistry}{College of Materials and Energy, South China Agricultural University}{2018 -- 2022}{Guangzhou, China}
\textsc{CGPA}: 3.93/5.00


\cvsection{Dissertation Work}
\cvevent{MSc Thesis}{Relativistic Density Functional Theory Calculations of Molecular Polarizabilities: Formulation and Implementation}{}{}
\begin{itemize}
\item Reformulated the CPHF/KS approach.
\item Extended the CPHF/KS approach to include relativistic two- and four-component methods for analytical (hyper)polarizability calculations of systems with heavy atoms.
\item Improved the CPHF/KS approach to support calculations of non-collinear xc functionals.
\item Identified three accurate and efficient solvers for the CPHF/KS equations: Krylov subspace, Newton-Krylov, and direct methods.
\item Developed a Python code for implementing the above methods in PySCF.
\item \textbf{Research paper of this work is under review in "Physics of Plasmas"}.
\end{itemize}


\cvevent{BSc Thesis}{DFT Study on the Mechanism of Conversion from Dihydroartemisinic Acid to Artemisinin}{}{}
\begin{itemize}
\item Proposed a plausible reaction path.
\item Identified the initial dominant conformations using xTB.
\item Computed the thermochemical data using Gaussian.
\item Determined the rate-determining step of the reaction to optimize conditions for drug production.
\end{itemize}


\cvsection{Certificates}

\begin{itemize}
\item CET-6 Certificate.
\item NCRE-2 Certificate.
\item the \textbf{Second Price} Award in the Preliminary Round of the 2019 "FLTRP·ETIC Cup" English Writing Contest.
\end{itemize}


\cvsection{Hobbies}

\cvachievement{\faGlobe}{Exploring Nature and the World}{Observing the nature of things}

\cvachievement{\faMusic
}{Music Enthusiast}{Enjoying quality music across various genres}

\cvachievement{\faBook}{Reading and Lifelong Learning}{Pursuing continuous learning and meaningful discussions}


\clearpage


\end{document}